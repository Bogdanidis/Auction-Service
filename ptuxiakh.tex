% TEMPLATE for Usenix papers, specifically to meet requirements of
%  USENIX '05
% originally a template for producing IEEE-format articles using LaTeX.
%   written by Matthew Ward, CS Department, Worcester Polytechnic Institute.
% adapted by David Beazley for his excellent SWIG paper in Proceedings,
%   Tcl 96
% turned into a smartass generic template by De Clarke, with thanks to
%   both the above pioneers
% use at your own risk.  Complaints to /dev/null.
% make it two column with no page numbering, default is 10 point

% Munged by Fred Douglis <douglis@research.att.com> 10/97 to separate
% the .sty file from the LaTeX source template, so that people can
% more easily include the .sty file into an existing document.  Also
% changed to more closely follow the style guidelines as represented
% by the Word sample file. 

% Note that since 2010, USENIX does not require endnotes. If you want
% foot of page notes, don't include the endnotes package in the 
% usepackage command, below.
\documentclass[letterpaper,twocolumn,10pt]{article}
\usepackage{usenix,epsfig,endnotes,url}
\begin{document}

%don't want date printed
\date{}

%make title bold and 14 pt font (Latex default is non-bold, 16 pt)
\title{\Large \bf Web-Based Auction Service Analyzation and Implementation}

\author{
{\rm Nikos Bogdanidis}\\
Computer Science Department, University of Crete
}

\maketitle

% Use the following at camera-ready time to suppress page numbers.
% Comment it out when you first submit the paper for review.
\thispagestyle{empty}


\subsection*{Abstract}
In these times, online web-based systems are getting increasingly popular by each day. Auction services are a crucial component in the online marketplace. Starting, we will by analyzing some already existing systems and the practices that are being implemented. Continuing, we will develop a similar service adjusted to fit Greece's marketplace. Tools that are being used include Google's Golang programming language for designing the REST API, UML diagrams in order to illustrate the architecture and more.

\section{Introduction}

A paragraph of text goes here.  Lots of text.  Plenty of interesting
text. \\

More fascinating text. Features\endnote{Remember to use endnotes, not footnotes!} galore, plethora of promises.\\

\section{Existing Online Auction Systems' Analysis}

We will cover some of the most popular auction platforms/systems that are currently available online. To begin with, we will see into two of the most popular websites eBay(1), eBid(2) according to ~\cite{Websites}. Next, it would be helpful to analyze a Greek Real Estate auction service, eauction.gr(4). Lastly, we will see two of the most popular auction service APIs that automatically configure and run an auction platform, BidJS(5) and Easy.Auction(6) for building entire platforms, according to ~\cite{AuctionServices}. 
\begin{itemize}
\item \textbf{Functional Requirements:}
\begin{itemize}
\item \emph{How does an auction work in each platform?}
    \begin{enumerate}
    \item Everyone can list an item for bidding. On eBay, auctions are open to bids for exactly 1, 3, 5, 7, or 10 days. When time is up, the high bidder wins. There is also a "Buy Now" preset price option which automatically ends the auction if used by one bidder. There is also an option for anonymous bidding in which information is hidden between parties.
    \item It works like eBay auction system, that is, the certified user who has been verified by his debit card credentials offers his product and determines a starting price, from that moment, it only remains to wait for the bids from buyers and the end of the auction. If the seller wants, he can even stipulate a fixed selling price and if someone bids it, he automatically gets the product. 
    \item Auction here takes place online and lasts for a few hours. It offers only an incremental price placing above the current price. When the auction concludes 1st bidder in the list gets the item.
    \item Auctioning with BidJS works just like the rest of the systems. It offers organizations full customization of an auction platform. Highest bidder takes the product, considering its above the reserve price set by the seller. It provides options like auto-bidding, and custom amount bidding.
    \item Just like BidJS.
    \end{enumerate}
\item \emph{What is being auctioned?}
    \begin{enumerate}
    \item On eBay, categories include: Collectibles and art, Electronics, Entertainment memorabilia, Fashion, Home and garden, Motors, Real Estate, Sporting goods, Toys and hobbies, Tickets and travel, Pet supplies, Specialty services, Baby essentials.
    \item On eBid, categories include: electronics, books, clothing, antiques, jewelry, toys, artwork, magazines and more.
    \item Its consists mainly by real estate based on Greece.
    \item About anything that belongs to the legal marketplace. It is the organization's decision what products there will be auctioned. Usually rel estate, industrial, art, auto and livestock products.
    \item Like BidJS in easy.auction the customer can list whatever he wants and customize platform and categories/subcategories as they please.
    \end{enumerate}
\item \emph{Who can take part in the auction?}
    \begin{enumerate}
    \item Everyone except buyers included in a seller's black list.
    \item Everyone.
    \item Only legally certified notaries.
    \item Auctioneers have complete control of who is able to bid on an auction.
    \item Like BidJS, auctioneers can choose who takes part.
    \end{enumerate}    
\item \emph{How many rounds does it consist of?}
    \begin{itemize}
    \item N/A
    \end{itemize}
\item \emph{How do users find particular auctions?}
    \begin{enumerate}
    \item Users can search for a particular item or a particular sellers name, can view items in categories and sub-categories, view recommended items based on previous purchases, browse daily sale items or browse popular items and/or brands. He can view results based on distance, price, time elapsed, time remaining, condition, and number of bids.
    \item Users can search for a particular item or a particular sellers name or for a particular shop, can view items in categories and sub-categories, browse Frontpage Featured Listings, Closing Listings, New Listings. He can view results based on distance, price, time elapsed, time remaining and condition.
    \item Users can search for listings using the sellers full name, the price of the items or the auction date.
    \item Similarly, using the search box users can search for a products name or ID, view categories or group by reviews.
    \item Users can search each category or subcategory, search using Lot number, view HOT items, listing name etc. Moreover, users can set up keyword alerts and receive an email if an item matches the entered criteria.
    \end{enumerate}  
\item \emph{How does a user bid?}
    \begin{enumerate}
    \item A user has to enter a bid that's at minimum the current highest bid, use auto-bidding or use 1-click bidding (bid one increment higher than the current bid) that is available. In the last case, the bid is always accepted. On the first case though, bid can be rejected if in between someone else has raised the price.
    \item A user has to enter a bid that's at minimum  one increment higher than the current bid. Then the system responds with acceptance or rejection.
    \item Users can bid only an incremental price placing above the current price.
    \item Users can use regular bidding option using a custom bid, Buy Now option if it is available by the seller, Make Offer option which makes a "Buy Now" offer to the seller and live webcast bidding as well.
    \item The Admin of the system can customize bidding increments, options, bidder authorization requirements etc. Bidders then have the option of auto-bidding, custom bidding or buy now like the other platforms.
    \end{enumerate} 
\item \emph{How does the system inform users about current winning bid of an auction?}
    \begin{enumerate}
    \item Users can set up alerts to be sent via email or mobile app when they are outbid. They can also view their bid history to observe if they are currently the highest bidder in an item.
    \item eBid shows the bid history and the highest bidder to all users that view the item. It also informs bidders of the result via email after bidding ends.
    \item It shows the particular users current position in the "race" (1st,2nd...).
    \item It informs the users using notifications in the web page in case they are outbid while also marking the product if they are the highest bidder.
    \item Bidding history is available to users, current winner in displayed and emails can be sent if a user it outbid.
    \end{enumerate} 
\item \emph{Does the system support auto-bidding?}
    \begin{enumerate}
    \item EBay supports auto-bidding. EBay's automatic bidding system is provided to the current high bidder in an auction listing. It requires the user to enter a maximum price. EBay will bid automatically on behalf of the high bidder against under-bidders until the bid overcomes the maximum price by one increment . 
    \item System supports proxy bidding, meaning bidders have the option to set a maximum price that they would be willing to pay for an item and then allow the computer system to bid for them by the bid increment until someone places a higher bid than their maximum.
    \item eauction.gr does not support auto-bidding.
    \item BidJS supports auto-bidding using a maximum price.
    \item Easy.auction supports auto-bidding.
    \end{enumerate} 
\item \emph{What about security?}
    \begin{enumerate}
    \item EBay provides users with reviews of a seller which can be used to determine if one is trustworthy. It also protects the use of credit cards through its online transaction services, and has money back guarantees.
    \item Standard user phone and card verification.
    \item Its pretty high considering you have to be authorized by the government to take part to an auction.
    \item Bidders require to have verified emails. billing addresses and phone numbers. Anti-snipping is available.
    \item Admins can set up  up-front credit card, email, phone number verification for bidders, sellers or both. Anti-snipping is available.
    \end{enumerate}    
\end{itemize}
\item \textbf{Non Functional Requirements:}
\begin{itemize} 
\item \emph{How many users are online each day?}
    \begin{enumerate}
    \item According to ~\cite{EbayEbidStatistics} As of January 2013, eBay receives more than 1,520,000 unique visitors every day.
    \item According to ~\cite{EbayEbidStatistics} eBid receives round 60,000 unique visitors every day.
    \item N/A
    \item According to BidJS statistics the have around 6 million active users in the past 12 months.
    \item N/A
    \end{enumerate}
\item \emph{How many auctions are active at any given time?}
    \begin{enumerate}
    \item On eBay there are more than 1.7 billion listings, though that includes the non-auction listings too.
    \item There are more than 3.8 million live listings.
    \item There are 125.000+ listings on eauction.gr but only a portion are active.
    \item Over 6000 auctions including 750.000 items in the past 12 months.
    \item N/A
    \end{enumerate} 
\item \emph{Average duration of an auction?}
    \begin{enumerate}
    \item On eBay the default auction option is 7 days, which can be considered as the average.
    \item On eBid auctions can last anywhere from a day to indefinitely until they are sold.
    \item The auction takes place in one day, usually hours.
    \item N/A
    \item N/A
    \end{enumerate}    
\item \emph{How many users take part in each auction?}
    \begin{enumerate}
    \item eBay's bidders range from 1 to 1000 at the most.
    \item eBid's bidders range from 1 to 100 at the most.
    \item eauction.gr's bidders range from 1 to 100 at the most.
    \item Considering the 5.5 million bids the last 12 months, we can assume roughly 10 bids per item.
    \item N/A
    \end{enumerate}  
\item \emph{Bids per second for an item?}
    \begin{itemize}
    \item N/A
    \end{itemize}
\item \emph{How about log keeping?}
    \begin{enumerate}
    \item eBay keeps log of all the bids on a particular item for the duration of the auction. For sellers bidders information is also available in the log.
    \item eBid keeps log of all biddings that is accessible to everyone.
    \item User's bidding log is available to him during the auction. 
    \item N/A
    \item N/A
    \end{enumerate}
\end{itemize}
\end{itemize}

{\footnotesize \bibliographystyle{acm}
\bibliography{sample}}


\theendnotes

\end{document}







